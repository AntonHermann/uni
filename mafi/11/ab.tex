% Dokumentenklasse
\documentclass[
% globale Schriftgröße
    10pt,
% setzt Absatzabstand hoch
    parskip=half-,
% Format
    paper=a4,
% lädt Sprachpakete
    english,ngerman,
    fleqn,
]{scrartcl}

% //////////////////// Pakete laden ////////////////////
\usepackage{amsmath}
% modifiziert amsmath
\usepackage{mathtools}
% mathematische Symbole, f"ur \ceckmarks
\usepackage{amssymb}
% f"ur proof
\usepackage{amsthm}
% f"ur \mathscr
\usepackage{mathrsfs}
\usepackage{latexsym}
% f"ur bessere Worttrennung
\usepackage{microtype}
% Spracheinstellung
\usepackage[ngerman]{babel}
% f"ur Quellcode
\usepackage{verbatim}
\usepackage{listings}
% f"ur Umlaute und Sonderzeichen in der Tex-Datei
\usepackage[utf8]{inputenc}
\usepackage{graphicx}
% f"ur Tabellen mit gleicher Spaltenbreite und automatischen Umbr"uchen
\usepackage{tabularx}
\usepackage{fullpage}
% f"ur multirow in tabulars
\usepackage{multirow}
\usepackage{rotate}
% um Farben zu benutzen, kann mehr als das Paket color
\usepackage[cmyk,table]{xcolor}
% Verlinkungen
\usepackage[
% farbige Schrift, statt farbiger Rahmen
    colorlinks,
% verlinkt im Abb.Verzeichnis Seitenzahl statt Bildunterschrift
    linktocpage,
% setzt Farbe der Links auf blau
    linkcolor=blue
]{hyperref}
% nur f"ur digitale Anwendungen, url = "http://www.example.com"
% f"ur Webadressen wie e-mail usw.: "\url{http://www.example.com}"
\usepackage{url}
% f"ur versch. Aufzählungezeichen wie z.B. a)
\usepackage{enumerate}
% folgt ein Leerzeichen nach einem \Befehl, wird es nicht verschluckt
\usepackage{xspace}
% f"ur das Durchstreichen u.a. in Matheformeln mit \cancel
\usepackage{cancel}
% f"ur \forloop und \whiledo
\usepackage{ifthen}

% //////////////////// Syntaxhervorhebung ////////////////////
\lstloadlanguages{Python, Haskell, [LaTeX]TeX, Java}
\lstset{%
% \scriptsize: die Fontgröße f"ur den Quelltext
   basicstyle=\footnotesize\ttfamily,
% legt Farbe der Box fest
   backgroundcolor = \color{bgcolour},
% automatische Umbr"uche nur nach Leerzeichen?
   breakatwhitespace=false,
% automatische Zeilenumbr"uche?
   breaklines=true,
% die "uberschrift steht oben, t = top
   captionpos=t,
% Formatierung der Kommentare
   commentstyle=\color{codeblue}\ttfamily,
% einfacher Rahmen um den Quelltext
   frame=single,
% Behalte Leerzeichen im Quelltext. N"utzlich, um
% Einr"uckungen zu erhalten (benötigt eventuell columns=flexible)
   keepspaces=true,
% Formatierung der Schl"usselwörter
   keywordstyle=\bfseries\ttfamily\color{codepurple},
% Wohin mit den Zeilennummern?
% mögliche Werte sind: none, left, right
   numbers=left,
% Formatierung der Zeilennummern
   numberstyle=\tiny\color{codegreen},
% Abstand zwischen Zeilennummern und Quelltext
   numbersep=5pt,
% nummeriert nur jede i-te Zeile
   stepnumber=1,
% Zeige alle Leerzeichen mit Hilfe von speziellen Unterstrichen;
% "uberschreibt 'showstringspaces'
   showspaces=false,
% unterstreiche Leerzeichen nur in Zeichenketten
   showstringspaces=false,
% Zeige Tabulatoren in Zeichenketten mit Hilfe von speziellen
% Unterstrichen
   showtabs=false,
   flexiblecolumns=false,
% Schrittweite zwischen Zeilennummern.
% Der Wert 1 bedeutet, dass jede Zeile nummeriert wird.
   tabsize=1,
% Formatierung der Zeichenketten.
   stringstyle=\color{orange}\ttfamily,
% leere Zeilen werden nicht nummeriert
   numberblanklines=false,
% Abstand zum linken Seitenrand
   xleftmargin=1.2em,
% Abstand zum rechten Seitenrand
   xrightmargin=0.4em,
   aboveskip=2ex,
}

\lstdefinestyle{py}{%
   language=Python,
}
\lstdefinestyle{hs}{%
   language=Haskell,
}
\lstdefinestyle{tex}{%
    language=[LaTeX]TeX,
% Um LaTex im Quelltext einzuf"ugen
    escapeinside={\%*}{*)},
% Hervorhebung der TeX-Schl"usselwörter
    texcsstyle=*\bfseries\color{blue},
    morekeywords={*,$,\{,\},\[,\],lstinputlisting,includegraphics,%$
        rowcolor,columncolor,listoffigures,lstlistoflistings,
        subsection,subsubsection,textcolor,tableofcontents,colorbox,
        fcolorbox,definecolor,cellcolor,url,linktocpage,subtitle,
        subject,maketitle,usetikzlibrary,node,path,addbibresource,
        printbibliography},
    numbers=none,
    numbersep=0pt,
    xleftmargin=0.4em,
}

\lstdefinestyle{java}{%
    language=Java,
    extendedchars=true,
}

% f"uge einen x64-Assembler Dialekt hinzu
\lstdefinelanguage[x64]{Assembler}
% basiert auf dem "x86masm" Dialekt
   [x86masm]{Assembler}
 % zusätzliche Schl"usselwörter
   {morekeywords={CDQE,CQO,CMPSQ,CMPXCHG16B,JRCXZ,LODSQ,MOVSXD, %
                  POPFQ,PUSHFQ,SCASQ,STOSQ,IRETQ,RDTSCP,SWAPGS, %
                  rax,rdx,rcx,rbx,rsi,rdi,rsp,rbp, %
                  r8,r8d,r8w,r8b,r9,r9d,r9w,r9b}
}

\lstdefinestyle{c}{
    language=c,
    extendedchars=true,
}

% //////////////////// eigene Anweisungen ////////////////////
% benötigt package xspace
\newcommand\FU{Freie Universit\ät Berlin\xspace}
\newcommand\gdw{g.\,d.\,w.\xspace}
\newcommand\oBdA{o.\,B.\,d.\,A.\xspace}
\newcommand\N{\mathbb{N}\xspace}
\newcommand\Q{\mathbb{Q}\xspace}
\newcommand\R{\mathbb{R}\xspace}
\newcommand\Z{\mathbb{Z}\xspace}
\newcommand\ohneNull{\ensuremath{\backslash\lbrace 0\rbrace}}% \{0}
\newcommand\ohne[1]{\ensuremath{\backslash\lbrace#1\rbrace}}% \{#1}
% Schreibt Befehl \dh in \dhALT um
\let\dhALT\dh
%renew "uberschreibt command \dh
\renewcommand\dh{d.\,h.\xspace}
\newcommand{\from}{\ensuremath{\colon}}
\newcommand{\floor}[1]{\lfloor{#1}\rfloor}
\newcommand{\ceil}[1]{\lceil{#1}\rceil}
\newcommand{\abbrev}[2]{\expandafter\newcommand\csname
                        #1\endcsname{#2\xspace}}
\newcommand{\cclasss}[2]{\abbrev{#1}{\textsf{#2}}}
\newcommand{\cclass}[1]{\cclasss{#1}{#1}}
\cclasss{ccP}{P}
\cclass{NP}
\cclass{LOGSPACE}
\cclass{NL}

% //////////////// mathematische Funktionen ////////////////////
\DeclareMathOperator{\True}{True}
\DeclareMathOperator{\False}{False}

% //////////////////// eigene Theoreme ////////////////////
\newtheorem{theorem}{Satz}
\newtheorem{corollary}[theorem]{Korollar}
\newtheorem{lemma}[theorem]{Lemma}
\newtheorem{observation}[theorem]{Beobachtung}
\newtheorem{definition}[theorem]{Definition}
\newtheorem{Literatur}[theorem]{Literatur}

% konfiguriert proof
\makeatletter
\newenvironment{Proof}[1][\proofname]{\par
  \pushQED{\qed}%
  \normalfont \topsep6\p@\@plus6\p@\relax
  \trivlist
  \item[\hskip\labelsep
        \bfseries
    #1\@addpunct{.}]\ignorespaces
}{%
  \popQED\endtrivlist\@endpefalse
}
\makeatother

% //////////////////// eigene Farben ////////////////////
\let\definecolor=\xdefinecolor
\definecolor{FUgreen}{RGB}{153,204,0}
\definecolor{FUblue}{RGB}{0,51,102}

\definecolor{middlegray}{rgb}{0.5,0.5,0.5}
\definecolor{lightgray}{rgb}{0.8,0.8,0.8}
\definecolor{orange}{rgb}{0.8,0.3,0.3}
\definecolor{azur}{rgb}{0,0.7,1}
\definecolor{yac}{rgb}{0.6,0.6,0.1}
\definecolor{Pink}{rgb}{1,0,0.6}

\definecolor{bgcolour}{rgb}{0.97,0.97,0.97}
\definecolor{codegreen}{rgb}{0,0.6,0}
\definecolor{codegray}{rgb}{0.35,0.35,0.35}
\definecolor{codepurple}{rgb}{0.58,0,0.82}
\definecolor{codeblue}{rgb}{0.4,0.5,1}

% //////////////////// eigene Einstellungen ////////////////////

% verhindert Einr"uckung der 1. Zeile eines Absatzes
\parindent 0pt
%
% <-- Name der "ubungsleitung eintragen
\newcommand{\tutor}{Serkan Süner}
% <-- Nummer im KVV nachschauen
\newcommand{\tutoriumNo}{4}
% <-- Nummer des "ubungszettels
\newcommand{\ubungNo}{11}
% <-- Name der Lehrveranstaltung eintragen
\newcommand{\veranstaltung}{Mafi 3}
% <-- z.B. SoSo 17, WiSe 17/18
\newcommand{\semester}{WiSe 17/18}
% <-- Hier die Namen eintragen
\newcommand{\studenten}{Merlin Joseph \& Anton Oehler}
% <-- Hier Anzahl der Aufgaben eintragen
\newcommand{\aufgNo}{4}

% /////////////////////// BEGIN DOKUMENT /////////////////////////
\begin{document}
% ////////////// Bepunktung //////////////
%linksb"undig
\makebox[\dimexpr\textwidth][l]{%
    \begin{minipage}{\linewidth}
        \newcounter{AufgNo}
        \setcounter{AufgNo}{\aufgNo}
        \stepcounter{AufgNo}   % AufgNo++
        \newcounter{zahl}
        \def\and{&\xspace}
        \renewcommand{\arraystretch}{1.3}\setlength{\tabcolsep}{1em}
        \begin{tabular}{*{\value{AufgNo}}{|c} |}
            \hline
             \setcounter{zahl}{1}
             \whiledo{\value{zahl} < \value{AufgNo}}{%\AufgNo
                 \thezahl\and\stepcounter{zahl}%
             } $\sum$ \\ \hline
             \setcounter{zahl}{1}
             \whiledo{\value{zahl} < \value{AufgNo}}{%\AufgNo
                 \phantom{X}\and\stepcounter{zahl}%
            } \phantom{X}\\ \hline
        \end{tabular}
    \end{minipage}
}
% ////////////// Daten //////////////
\begin{center}

{\Large \veranstaltung, \semester}\par
{\large Tutor\_in: \tutor, Tutorium \tutoriumNo}\par
{\Large \"Ubung \ubungNo}\par
{\large \studenten}\par
\today
\end{center}
\vspace{-3ex}             % Abstand
\rule{\linewidth}{0.8pt}  % horizontale Linie

\newcommand{\dx}{\> dx}
\newcommand{\dt}{\> dt}
\newcommand{\du}{\> du}
% /////////////////////// Aufgabe 1 /////////////////////////
Anmerkung: Der Einfachheit halber wird auf die Notation der Konstante $c$ in
den Zwischenschritten verzichtet.


\section*{Aufgabe 2: partielle Integration}

\begin{enumerate}[a)]
\item \[ \int x^3 \ln 2x \dx \]
    \[ u'(x) = x^3; \quad u(x) = \dfrac{1}{4} x^4 \]
    \[ v(x)  = \ln 2x; \quad v'(x) = \dfrac{1}{2x} \cdot 2 = \dfrac{1}{x}\]
    Partielle Integration:
    \begin{align*}
        \int u'(x) \cdot v(x) \dx &=
            u(x)\cdot v(x) - \int u(x) \cdot v'(x) \dx\\
        \int x^3 \cdot \ln 2x \dx &=
            \dfrac{1}{4} x^4 \cdot \ln 2x -
            \int \dfrac{1}{4} x^4 \cdot \dfrac{1}{x} \dx\\
        &= \dfrac{x^4}{4} \cdot \ln 2x - \int \dfrac{x^3}{4}\\
        &= \dfrac{x^4}{4} \cdot \ln 2x - \dfrac{1}{4} \cdot \dfrac{x^4}{4}\\
        &= \left(\ln 2x - \dfrac{1}{4}\right) \cdot \dfrac{x^4}{4} + c
    \end{align*}
\item \[ \int x^2 e^{2x} \dx \]
    \[ u'(x) = e^{2x}; \quad u(x) = \dfrac{1}{2} e^{2x} \]
    \[ v(x) = x^2; \quad v'(x) = 2x \]
    Partielle Integration:
    \begin{align*}
        \int u'(x) \cdot v(x) \dx &=
         u(x)\cdot v(x) - \int u(x) \cdot v'(x) \dx\\
        \int e^{2x} x^2 \dx &= \dfrac{1}{2} e^{2x} \cdot x^2 -
            \int \dfrac{1}{2} e^{2x} \cdot 2x \dx\\
        &= \dfrac{1}{2} e^{2x} x^2 - \int e^{2x}x \dx\\
        &\text{erneute partielle Integration:}\\
        &= \dfrac{1}{2}e^{2x}x^2-\left(\dfrac{1}{2}e^{2x}x-\int\dfrac{1}{2}e^{2x}\dx\right)\\
        &= \dfrac{1}{2}e^{2x}x^2-\left(\dfrac{1}{2}e^{2x}x-\dfrac{1}{4}e^{2x}\right)\\
        &= \dfrac{1}{2}e^{2x}x^2-\dfrac{1}{2}e^{2x}x+\dfrac{1}{4}e^{2x}\\
        &= \dfrac{1}{2}e^{2x} \cdot \left(x^2 - x + \dfrac{1}{2}\right) + c
    \end{align*}
\item \[ \int \sin (\ln x) \dx = \int 1 \cdot \sin (\ln x) \dx\]
    \[ u'(x) = 1; \quad u(x) = x \]
    \[ v(x) = \sin(\ln x); \quad v'(x) = \dfrac{\cos(\ln x)}{x} \]
    Partielle Integration:
    \begin{align*}
        \int u'(x) \cdot v(x) \dx &=
            u(x)\cdot v(x) - \int u(x) \cdot v'(x) \dx\\
        \int 1 \cdot \sin (\ln x) \dx\
        &= x\cdot\sin(\ln x) -\int x\cdot \dfrac{\cos(\ln x)}{x} \dx\\
        &= x\cdot\sin(\ln x) -\int \cos(\ln x) \dx\\
        &= x\cdot\sin(\ln x) -\left(x\cdot\cos(\ln x)-\int-\sin(\ln x)\dx\right)\\
        &= x\cdot\sin(\ln x) -\left(x\cdot\cos(\ln x)+\int\sin(\ln x)\dx\right)\\
        &= x\cdot\sin(\ln x) + x\cdot\cos(\ln x)-\int\sin(\ln x)\dx
            &|\>+\int\sin(\ln x)\dx\\
        2\cdot\int\sin(\ln x)\dx &= x\cdot\left(\sin(\ln x)+\cos(\ln x)\right)\\
        \int\sin(\ln x)\dx
            &= \dfrac{x\cdot\left(\sin(\ln x)+\cos(\ln x)\right)}{2} + c
    \end{align*}
\item \[ \int e^x \cos 2x \dx \]
    \[ u'(x) = u(x) = e^x \]
    \[ v(x) = \cos 2x; \quad v'(x) = -2\sin 2x \]
    \begin{align*}
        \int u'(x) \cdot v(x) \dx &=
            u(x)\cdot v(x) - \int u(x) \cdot v'(x) \dx\\
        \int e^x \cos 2x \dx
        &= e^x \cdot\cos 2x - \int -e^x \cdot 2\sin 2x \dx\\
        &= e^x \cdot\cos 2x + 2\int e^x \cdot \sin 2x \dx\\
        &= e^x \cdot\cos 2x + 2\left(e^x \cdot \sin 2x -
            \int e^x\cdot 2 \cos 2x \dx \right)\\
        &= e^x \cdot\cos 2x + 2\left(e^x \cdot \sin 2x -
            2 \int e^x\cdot \cos 2x \dx \right)\\
        &= e^x \cdot\cos 2x + 2 e^x \cdot \sin 2x -
            4 \int e^x\cdot \cos 2x \dx &|\>+4\int e^x\cdot\cos 2x \dx\\
        5 \int e^x \cos 2x \dx &= e^x \left(\cos 2x + 2 \sin 2x \right)\\
        \int e^x \cos 2x \dx &= \dfrac{e^x \left(\cos 2x + 2\sin 2x \right)}{5} +
        c
    \end{align*}
\end{enumerate}

\section*{Aufgabe 3}
\begin{enumerate}
\item \[ \int \dfrac{2}{(3x-4)^3} \dx \]
    Substitution mit $t = 3x-4$ und $3\dx = \dt$ ($\dx = \frac{\dt}{3}$)
    \begin{align*}
        \int \dfrac{2}{(3x-4)^3} \dx
        &= \int \dfrac{2}{3t^3} \dt \\
        &= \dfrac{2}{3} \int t^{-3} \dt\\
        &= \dfrac{2}{3} \cdot \dfrac{-1}{2} \cdot t^{-2} + c\\
        &= - \dfrac{1}{3t^2} + c\\
        &\text{Resubstitution:}\\
        &= - \dfrac{1}{3(3x-4)^2} + c
    \end{align*}
\item \[ \int \dfrac{dx}{x^3 + 3x^2 + 3x + 1} = \int \dfrac{dx}{(x+1)^3}\]
    Substitution mit $t = x+1$ und $1dx = dt$
    \begin{align*}
        \int \dfrac{dx}{(x+1)^3}
        &= \int \dfrac{dt}{t^3} = \int t^{-3} \dt\\
        &= -\dfrac{1}{2} \cdot t^{-2}\\
        &= -\dfrac{1}{2t^2}\\
        &\text{Resubstitution:}\\
        &= -\dfrac{1}{2(x+1)^2} + c\\
    \end{align*}
\item \[ \int \dfrac{6x}{x^4+6x^2+9} \dx = 6 \int \dfrac{x}{(x^2+3)^2} \dx\]
    Substitution mit $t = \dfrac{x^2}{2}$ und $x\dx = dt$
    \[ 6 \int \dfrac{x}{(x^2+3)^2}\dx = 6 \int \dfrac{dt}{(2t+3)^2} \]
    Substitution mit $u = 2t+3$ und $2\dt = du$ ($\dt = \frac{\du}{2}$)
        \begin{align*}
            6 \int \dfrac{dt}{(2t+3)^2}
            &= 6 \int \dfrac{du}{2u^2}\\
            &= 3 \int u^{-2} \du\\
            &= 3 \cdot \dfrac{-1}{2u} = -\dfrac{3}{2u}\\
            &\text{Resubstitution:}\\
            &= -\dfrac{3}{2(2t+3)}\\
            &= -\dfrac{3}{(4t+6)}\\
            &\text{Resubstitution:}\\
            &= -\dfrac{3}{\left(4\left(\dfrac{x^2}{2}\right)+6\right)}\\
            &= -\dfrac{3}{2x^2+6} + c\\
        \end{align*}
\end{enumerate}



% /////////////////////// END DOKUMENT /////////////////////////
\end{document}

