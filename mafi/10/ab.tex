% Dokumentenklasse
\documentclass[
% globale Schriftgröße
    10pt,
% setzt Absatzabstand hoch
    parskip=half-,
% Format
    paper=a4,
% lädt Sprachpakete
    english,ngerman,
% left align equations
    fleqn
]{scrartcl}

% //////////////////// Pakete laden ////////////////////
\usepackage{amsmath}
% modifiziert amsmath
\usepackage{mathtools}
% mathematische Symbole, f"ur \ceckmarks
\usepackage{amssymb}
% f"ur proof
\usepackage{amsthm}
% f"ur \mathscr
\usepackage{mathrsfs}
\usepackage{latexsym}
% f"ur bessere Worttrennung
\usepackage{microtype}
% Spracheinstellung
\usepackage[ngerman]{babel}
% f"ur Quellcode
\usepackage{verbatim}
\usepackage{listings}
% f"ur Umlaute und Sonderzeichen in der Tex-Datei
\usepackage[utf8]{inputenc}
\usepackage{graphicx}
% f"ur Tabellen mit gleicher Spaltenbreite und automatischen Umbr"uchen
\usepackage{tabularx}
\usepackage{fullpage}
% f"ur multirow in tabulars
\usepackage{multirow}
\usepackage{rotate}
% um Farben zu benutzen, kann mehr als das Paket color
\usepackage[cmyk,table]{xcolor}
% Verlinkungen
\usepackage[
% farbige Schrift, statt farbiger Rahmen
    colorlinks,
% verlinkt im Abb.Verzeichnis Seitenzahl statt Bildunterschrift
    linktocpage,
% setzt Farbe der Links auf blau
    linkcolor=blue
]{hyperref}
% nur f"ur digitale Anwendungen, url = "http://www.example.com"
% f"ur Webadressen wie e-mail usw.: "\url{http://www.example.com}"
\usepackage{url}
% f"ur versch. Aufzählungezeichen wie z.B. a)
\usepackage{enumerate}
% folgt ein Leerzeichen nach einem \Befehl, wird es nicht verschluckt
\usepackage{xspace}
% f"ur das Durchstreichen u.a. in Matheformeln mit \cancel
\usepackage{cancel}
% f"ur \forloop und \whiledo
\usepackage{ifthen}

% //////////////////// Syntaxhervorhebung ////////////////////
\lstloadlanguages{Python, Haskell, [LaTeX]TeX, Java}
\lstset{%
% \scriptsize: die Fontgröße f"ur den Quelltext
   basicstyle=\footnotesize\ttfamily,
% legt Farbe der Box fest
   backgroundcolor = \color{bgcolour},
% automatische Umbr"uche nur nach Leerzeichen?
   breakatwhitespace=false,
% automatische Zeilenumbr"uche?
   breaklines=true,
% die "uberschrift steht oben, t = top
   captionpos=t,
% Formatierung der Kommentare
   commentstyle=\color{codeblue}\ttfamily,
% einfacher Rahmen um den Quelltext
   frame=single,
% Behalte Leerzeichen im Quelltext. N"utzlich, um
% Einr"uckungen zu erhalten (benötigt eventuell columns=flexible)
   keepspaces=true,
% Formatierung der Schl"usselwörter
   keywordstyle=\bfseries\ttfamily\color{codepurple},
% Wohin mit den Zeilennummern?
% mögliche Werte sind: none, left, right
   numbers=left,
% Formatierung der Zeilennummern
   numberstyle=\tiny\color{codegreen},
% Abstand zwischen Zeilennummern und Quelltext
   numbersep=5pt,
% nummeriert nur jede i-te Zeile
   stepnumber=1,
% Zeige alle Leerzeichen mit Hilfe von speziellen Unterstrichen;
% "uberschreibt 'showstringspaces'
   showspaces=false,
% unterstreiche Leerzeichen nur in Zeichenketten
   showstringspaces=false,
% Zeige Tabulatoren in Zeichenketten mit Hilfe von speziellen
% Unterstrichen
   showtabs=false,
   flexiblecolumns=false,
% Schrittweite zwischen Zeilennummern.
% Der Wert 1 bedeutet, dass jede Zeile nummeriert wird.
   tabsize=1,
% Formatierung der Zeichenketten.
   stringstyle=\color{orange}\ttfamily,
% leere Zeilen werden nicht nummeriert
   numberblanklines=false,
% Abstand zum linken Seitenrand
   xleftmargin=1.2em,
% Abstand zum rechten Seitenrand
   xrightmargin=0.4em,
   aboveskip=2ex,
}

\lstdefinestyle{py}{%
   language=Python,
}
\lstdefinestyle{hs}{%
   language=Haskell,
}
\lstdefinestyle{tex}{%
    language=[LaTeX]TeX,
% Um LaTex im Quelltext einzuf"ugen
    escapeinside={\%*}{*)},
% Hervorhebung der TeX-Schl"usselwörter
    texcsstyle=*\bfseries\color{blue},
    morekeywords={*,$,\{,\},\[,\],lstinputlisting,includegraphics,%$
        rowcolor,columncolor,listoffigures,lstlistoflistings,
        subsection,subsubsection,textcolor,tableofcontents,colorbox,
        fcolorbox,definecolor,cellcolor,url,linktocpage,subtitle,
        subject,maketitle,usetikzlibrary,node,path,addbibresource,
        printbibliography},
    numbers=none,
    numbersep=0pt,
    xleftmargin=0.4em,
}

\lstdefinestyle{java}{%
    language=Java,
    extendedchars=true,
}

% f"uge einen x64-Assembler Dialekt hinzu
\lstdefinelanguage[x64]{Assembler}
% basiert auf dem "x86masm" Dialekt
   [x86masm]{Assembler}
 % zusätzliche Schl"usselwörter
   {morekeywords={CDQE,CQO,CMPSQ,CMPXCHG16B,JRCXZ,LODSQ,MOVSXD, %
                  POPFQ,PUSHFQ,SCASQ,STOSQ,IRETQ,RDTSCP,SWAPGS, %
                  rax,rdx,rcx,rbx,rsi,rdi,rsp,rbp, %
                  r8,r8d,r8w,r8b,r9,r9d,r9w,r9b}
}

\lstdefinestyle{c}{
    language=c,
    extendedchars=true,
}

% //////////////////// eigene Anweisungen ////////////////////
% benötigt package xspace
\newcommand\FU{Freie Universit\ät Berlin\xspace}
\newcommand\gdw{g.\,d.\,w.\xspace}
\newcommand\oBdA{o.\,B.\,d.\,A.\xspace}
\newcommand\N{\mathbb{N}\xspace}
\newcommand\Q{\mathbb{Q}\xspace}
\newcommand\R{\mathbb{R}\xspace}
\newcommand\Z{\mathbb{Z}\xspace}
\newcommand\ohneNull{\ensuremath{\backslash\lbrace 0\rbrace}}% \{0}
\newcommand\ohne[1]{\ensuremath{\backslash\lbrace#1\rbrace}}% \{#1}
% Schreibt Befehl \dh in \dhALT um
\let\dhALT\dh
%renew "uberschreibt command \dh
\renewcommand\dh{d.\,h.\xspace}
\newcommand{\from}{\ensuremath{\colon}}
\newcommand{\floor}[1]{\lfloor{#1}\rfloor}
\newcommand{\ceil}[1]{\lceil{#1}\rceil}
\newcommand{\abbrev}[2]{\expandafter\newcommand\csname
                        #1\endcsname{#2\xspace}}
\newcommand{\cclasss}[2]{\abbrev{#1}{\textsf{#2}}}
\newcommand{\cclass}[1]{\cclasss{#1}{#1}}
\cclasss{ccP}{P}
\cclass{NP}
\cclass{LOGSPACE}
\cclass{NL}

% //////////////// mathematische Funktionen ////////////////////
\DeclareMathOperator{\True}{True}
\DeclareMathOperator{\False}{False}

% //////////////////// eigene Theoreme ////////////////////
\newtheorem{theorem}{Satz}
\newtheorem{corollary}[theorem]{Korollar}
\newtheorem{lemma}[theorem]{Lemma}
\newtheorem{observation}[theorem]{Beobachtung}
\newtheorem{definition}[theorem]{Definition}
\newtheorem{Literatur}[theorem]{Literatur}

% konfiguriert proof
\makeatletter
\newenvironment{Proof}[1][\proofname]{\par
  \pushQED{\qed}%
  \normalfont \topsep6\p@\@plus6\p@\relax
  \trivlist
  \item[\hskip\labelsep
        \bfseries
    #1\@addpunct{.}]\ignorespaces
}{%
  \popQED\endtrivlist\@endpefalse
}
\makeatother

% //////////////////// eigene Farben ////////////////////
\let\definecolor=\xdefinecolor
\definecolor{FUgreen}{RGB}{153,204,0}
\definecolor{FUblue}{RGB}{0,51,102}

\definecolor{middlegray}{rgb}{0.5,0.5,0.5}
\definecolor{lightgray}{rgb}{0.8,0.8,0.8}
\definecolor{orange}{rgb}{0.8,0.3,0.3}
\definecolor{azur}{rgb}{0,0.7,1}
\definecolor{yac}{rgb}{0.6,0.6,0.1}
\definecolor{Pink}{rgb}{1,0,0.6}

\definecolor{bgcolour}{rgb}{0.97,0.97,0.97}
\definecolor{codegreen}{rgb}{0,0.6,0}
\definecolor{codegray}{rgb}{0.35,0.35,0.35}
\definecolor{codepurple}{rgb}{0.58,0,0.82}
\definecolor{codeblue}{rgb}{0.4,0.5,1}

% //////////////////// eigene Einstellungen ////////////////////

% verhindert Einr"uckung der 1. Zeile eines Absatzes
\parindent 0pt
%
% <-- Name der "ubungsleitung eintragen
\newcommand{\tutor}{Serkan Süner}
% <-- Nummer im KVV nachschauen
\newcommand{\tutoriumNo}{4}
% <-- Nummer des "ubungszettels
\newcommand{\ubungNo}{10}
% <-- Name der Lehrveranstaltung eintragen
\newcommand{\veranstaltung}{Mafi 3}
% <-- z.B. SoSo 17, WiSe 17/18
\newcommand{\semester}{WiSe 17/18}
% <-- Hier die Namen eintragen
\newcommand{\studenten}{Merlin Joseph \& Anton Oehler}
% <-- Hier Anzahl der Aufgaben eintragen
\newcommand{\aufgNo}{4}

% /////////////////////// BEGIN DOKUMENT /////////////////////////
\begin{document}
% ////////////// Bepunktung //////////////
%linksb"undig
\makebox[\dimexpr\textwidth][l]{%
    \begin{minipage}{\linewidth}
        \newcounter{AufgNo}
        \setcounter{AufgNo}{\aufgNo}
        \stepcounter{AufgNo}   % AufgNo++
        \newcounter{zahl}
        \def\and{&\xspace}
        \renewcommand{\arraystretch}{1.3}\setlength{\tabcolsep}{1em}
        \begin{tabular}{*{\value{AufgNo}}{|c} |}
            \hline
             \setcounter{zahl}{1}
             \whiledo{\value{zahl} < \value{AufgNo}}{%\AufgNo
                 \thezahl\and\stepcounter{zahl}%
             } $\sum$ \\ \hline
             \setcounter{zahl}{1}
             \whiledo{\value{zahl} < \value{AufgNo}}{%\AufgNo
                 \phantom{X}\and\stepcounter{zahl}%
            } \phantom{X}\\ \hline
        \end{tabular}
    \end{minipage}
}
% ////////////// Daten //////////////
\begin{center}

{\Large \veranstaltung, \semester}\par
{\large Tutor\_in: \tutor, Tutorium \tutoriumNo}\par
{\Large \"Ubung \ubungNo}\par
{\large \studenten}\par
\today
\end{center}
\vspace{-3ex}             % Abstand
\rule{\linewidth}{0.8pt}  % horizontale Linie

% /////////////////////// Aufgabe 1 /////////////////////////
\section*{Aufgabe 1: Kurvendiskussion}
\begin{align*}
f(x)  & = x^2e^x\\
u(x)  & = x^2; u'(x) = 2x\\
v(x)  & = e^x; v'(x) = e^x\\
f(x)  & = u(x) \cdot v(x)\\
f'(x) & = u'(x) \cdot v(x) + u(x) \cdot v'(x)\\
      & = 2x \cdot e^x + x^2 \cdot e^x = e^x(2x+x^2)\\
w(x)  & = 2x+x^2; w'(x) = 2+2x\\
f'(x) & = v(x) \cdot w(x)\\
f"(x) & = v'(x) \cdot w(x) + v(x) \cdot w'(x)\\
f"(x) & = e^x \cdot (2x+x^2) + e^x \cdot 2+2x\\
      & = e^x \cdot (2x+x^2+2+2x) = e^x \cdot (x^2 + 4x + 2)
\end{align*}

Bestimmung der Extrema:
\begin{align*}
    0 & = f'(x)\\
      & = e^x(2x+x^2)\\
e^x&\text{ kann nicht 0 werden, also:}\\
    0 & = 2x+x^2\\
    0 & = x \cdot (2+x)\\
    x_1 & = 0\\
    x_2 & = -1
\end{align*}
In $f"(x)$ einsetzen, um die Art der Extrema zu ermitteln:\\
$f"(0) = e^0 \cdot 2 = 2 \Rightarrow$ Minimum\\
$f"(-1) = e^{-1} \cdot (1 -4 + 2) = -\dfrac{1}{e} \Rightarrow$ Maximum\\
In $f(x)$ einsetzen, um die zugehoerigen y-Koordinaten zu ermitteln:\\
$f(0) = 0^2e^0 = 0 \Rightarrow Min: (0; 0)$\\
$f(-1) = (-1)^2 e^{-1} = \dfrac{1}{e} \Rightarrow Max: (-1; \dfrac{1}{e})$\\

Ermittlung der Wendepunkte:
\begin{align*}
    0 & = f"(x)\\
      & = e^x (x^2 + 4x + 2)\\
e^x&\text{ kann nicht 0 werden, also:}\\
    0 & = x^2 + 4x + 2\\
    x_{1,2} & = -2 \pm \sqrt{4-2} = -2 \pm \sqrt{2}\\
    x_1 & = -2 - \sqrt{2}\\
    x_2 & = -2 + \sqrt{2}\\
\end{align*}

Ermittlung der zugehoerigen y-Koordinaten:\\
$f(-2-\sqrt{2}) = (6+4\sqrt{2})e^{-2-\sqrt{2}} \Rightarrow W_1(-2-\sqrt{2};(6+4\sqrt{2})e^{-2-\sqrt{2}})$\\
$f(-2+\sqrt{2}) = (6-4\sqrt{2})e^{-2+\sqrt{2}} \Rightarrow W_2(-2+\sqrt{2};(6-4\sqrt{2})e^{-2+\sqrt{2}})$\\

\section*{Aufgabe 3: Regel von Bernoulli-L`Hospital}
\newcommand{\limZ}{\lim_{x\to 0}}
\newcommand{\limTo}[1]{\lim_{x\to#1}}
\newcommand{\bern}{\Rightarrow\text{Bernoulli-L'Hostpital-Kriterium}}
\begin{enumerate}[a)]
\item
    \begin{align*}
        &\lim_{x\to 0} \dfrac{\cos 2x-\cos x}{x^2}\\
        &\lim_{x\to 0} \cos 2x-\cos x = 1-1 = 0\\
        &\lim_{x\to 0} x^2 = 0^2 = 0\\
        \Rightarrow &\lim_{x\to 0} \dfrac{0}{0} \bern\\
        &\lim_{x\to 0} \dfrac{u(x)}{v(x)} = \limZ \dfrac{u'(x)}{v'(x)}\\
        &u(x) = \cos 2x-\cos x; u'(x) = \sin x-2\sin(2x)\\
        &v(x) = x^2; v'(x) = 2x\\
        &\limZ \dfrac{u'(x)}{v'(x)}
            = \limZ \dfrac{\sin x-2\sin(2x)}{2x} = \limZ \dfrac{0}{0} \bern\\
        &\limZ \dfrac{u'(x)}{v'(x)} = \limZ \dfrac{u"(x)}{v"(x)}\\
        &u'(x) = \sin x - 2\sin(2x); u"(x) = \cos x - 4\cos(2x)\\
        &v'(x) = 2x; v"(x) = 2\\
        &\limZ \dfrac{u"(x)}{v"(x)} = \dfrac{1-4}{2} = -\dfrac{3}{2}\\
    \end{align*}
\item
    \begin{align*}
        &\limTo{1} \dfrac{e^{x^2}-e^x}{x^2-1}\\
        &\limTo{1} e^{x^2}-e^x = e^1-e^1 = 0\\
        \Rightarrow &\limTo{1} \dfrac{0}{0} \bern\\
        &\limTo{1} \dfrac{u(x)}{v(x)} = \limTo{1} \dfrac{u'(x)}{v'(x)}\\
        &u(x) = e^{x^2}-e^x; u'(x) = 2xe^{x^2}-e^x\\
        &v(x) = x^2-1; v'(x) = 2x\\
        &\limTo{1} \dfrac{u'(x)}{v'(x)} = \limTo{1} \dfrac{2xe^{x^2}-e^x}{2x}
            = \dfrac{e}{2}\\
    \end{align*}
\item
    \begin{align*}
        &\limTo{0+}\left(\dfrac{1}{x} - \dfrac{1}{\sin x}\right)
            = \limTo{0+}\dfrac{\sin x - x}{x\cdot\sin x}
            = \limTo{0+}\dfrac{0}{0} \bern\\
        &\limTo{0+} \dfrac{u(x)}{v(x)} = \limTo{0+} \dfrac{u'(x)}{v'(x)}\\
        &u(x) = \sin x - x; u'(x) = \cos x - 1\\
        &v(x) = x \cdot \sin x; v'(x) = \sin x + x \cdot \cos x\\
        &\limTo{0+} \dfrac{u'(x)}{v'(x)}
            = \limTo{0+} \dfrac{\cos x - 1}{\sin x + x \cdot \cos x}
            = \dfrac{1-1}{0+0} = \dfrac{0}{0} \bern\\
        &\limTo{0+} \dfrac{u'(x)}{v'(x)} = \limTo{0+} \dfrac{u"(x)}{v"(x)}\\
        &u'(x) = \cos x - 1; u"(x) = -\sin x\\
        &v'(x) = \sin x + x \cdot \cos x; v"(x) = 2\cos x - x \cdot \sin x\\
        &\limTo{0+} \dfrac{u"(x)}{v"(x)}
            = \limTo{0+}\dfrac{-\sin x}{2\cos x - x \cdot \sin x}
            = \dfrac{0}{2-0} = 0\\
    \end{align*}
\item
    \begin{align*}
        &\limTo{\infty}\dfrac{\ln x}{\ln(x^2+1)}=\dfrac{\infty}{\infty} \bern\\
        &\limTo{\infty}\dfrac{u(x)}{v(x)}=\limTo{\infty}\dfrac{u'(x)}{v'(x)}\\
        &u(x) = \ln x; u'(x) = \dfrac{1}{x}\\
        &v(x) = \ln(x^2 + 1); v'(x) = \dfrac{2x}{x^2+1}\\
        &\limTo{\infty} \dfrac{u'(x)}{v'(x)}
            = \limTo{\infty} \dfrac{ \dfrac{1}{x} }{ \dfrac{2x}{x^2+1} }
            = \limTo{\infty} \dfrac{x^2+1}{2x^2} = \dfrac{\infty}{\infty}\bern\\
        &\limTo{\infty} \dfrac{u_2(x)}{v_2(x)}
            =\limTo{\infty} \dfrac{u_2'(x)}{v_2'(x)}\\
        &u_2(x) = x^2 + 1; u_2'(x) = 2x\\
        &v_2(x) = 2x^2; v_2'(x) = 4x\\
        &\limTo{\infty}\dfrac{u_2'(x)}{v_2'(x)} = \limTo{\infty}\dfrac{2x}{4x}
            = \limTo{\infty} \dfrac{2}{4} = \dfrac{2}{4}\\
    \end{align*}
\end{enumerate}

\section*{Aufgabe 4: Umkehrfunktionen}
\begin{enumerate}[a)]
\item
    \begin{align*}
        f(x) &= 2x^2 - 8x + 2\\
             &= 2(x^2 - 4x + 1)\\
             &= 2(x^2 - 2\cdot2x + 2^2 - 2^2 + 1)\\
             &= 2((x-2)^2 - 3)\\
             &= 2(x-2)^2 - 6
    \end{align*}
    $f(x)$ ist umkehrbar im Intervall $[a, \infty)$ mit $a = 2$.\\
    Umkehrung: $g = f^{-1}$
    \begin{align*}
        y &= 2(x-2)^2 - 6 &| +6\\
        y + 6 &= 2(x-2)^2 &| \cdot\dfrac{1}{2}\\
        \dfrac{y+6}{2} &= (x-2)^2 &|\sqrt{}\\
        \sqrt{\dfrac{y+6}{2}} &= x - 2 &| +2\\
        \sqrt{\dfrac{y+6}{2}} + 2 &= x = g(y)\\
    \end{align*}
\item
    \begin{align*}
        g(x) &= \sqrt{\dfrac{x+6}{2}} + 2\\
             &= u(v(x)) + 2\\
        u(x) &= \sqrt{x}; u'(x) = \dfrac{1}{2\sqrt{x}}\\
        v(x) &= \dfrac{x+6}{2}; v'(x) = \dfrac{1}{2}\\
        g'(x) &= u'(v(x)) \cdot v'(x)\\
            &= \dfrac{1}{2\sqrt{\dfrac{x+6}{2}}} \cdot \dfrac{1}{2}\\
            &= \dfrac{1}{4\sqrt{\dfrac{x+6}{2}}}\\
            &= \dfrac{ \sqrt{\dfrac{x+6}{2}} }
                { 4 \left( \dfrac{x+6}{2} \right)}
            = \dfrac{\sqrt{\dfrac{x+6}{2}}}{2x+12}
    \end{align*}
\end{enumerate}

% /////////////////////// END DOKUMENT /////////////////////////
\end{document}
