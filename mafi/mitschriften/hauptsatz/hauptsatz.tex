% Dokumentenklasse
\documentclass[
% globale Schriftgröße
    10pt,
% setzt Absatzabstand hoch
    parskip=half-,
% Format
    paper=a4,
% lädt Sprachpakete
    english,ngerman,
]{scrartcl}

% //////////////////// Pakete laden ////////////////////
\usepackage{amsmath}
% modifiziert amsmath
\usepackage{mathtools}
% mathematische Symbole, f"ur \ceckmarks
\usepackage{amssymb}
% f"ur proof
\usepackage{amsthm}
% f"ur \mathscr
\usepackage{mathrsfs}
\usepackage{latexsym}
% f"ur bessere Worttrennung
\usepackage{microtype}
% Spracheinstellung
\usepackage[ngerman]{babel}
% f"ur Quellcode
\usepackage{verbatim}
\usepackage{listings}
% f"ur Umlaute und Sonderzeichen in der Tex-Datei
\usepackage[utf8]{inputenc}
\usepackage{graphicx}
% f"ur Tabellen mit gleicher Spaltenbreite und automatischen Umbr"uchen
\usepackage{tabularx}
\usepackage{fullpage}
% f"ur multirow in tabulars
\usepackage{multirow}
\usepackage{rotate}
% um Farben zu benutzen, kann mehr als das Paket color
\usepackage[cmyk,table]{xcolor}
% Verlinkungen
\usepackage[
% farbige Schrift, statt farbiger Rahmen
    colorlinks,
% verlinkt im Abb.Verzeichnis Seitenzahl statt Bildunterschrift
    linktocpage,
% setzt Farbe der Links auf blau
    linkcolor=blue
]{hyperref}
% nur f"ur digitale Anwendungen, url = "http://www.example.com"
% f"ur Webadressen wie e-mail usw.: "\url{http://www.example.com}"
\usepackage{url}
% f"ur versch. Aufzählungezeichen wie z.B. a)
\usepackage{enumerate}
% folgt ein Leerzeichen nach einem \Befehl, wird es nicht verschluckt
\usepackage{xspace}
% f"ur das Durchstreichen u.a. in Matheformeln mit \cancel
\usepackage{cancel}
% f"ur \forloop und \whiledo
\usepackage{ifthen}

% //////////////////// Syntaxhervorhebung ////////////////////
\lstloadlanguages{Python, Haskell, [LaTeX]TeX, Java}
\lstset{%
% \scriptsize: die Fontgröße f"ur den Quelltext
   basicstyle=\footnotesize\ttfamily,
% legt Farbe der Box fest
   backgroundcolor = \color{bgcolour},
% automatische Umbr"uche nur nach Leerzeichen?
   breakatwhitespace=false,
% automatische Zeilenumbr"uche?
   breaklines=true,
% die "uberschrift steht oben, t = top
   captionpos=t,
% Formatierung der Kommentare
   commentstyle=\color{codeblue}\ttfamily,
% einfacher Rahmen um den Quelltext
   frame=single,
% Behalte Leerzeichen im Quelltext. N"utzlich, um
% Einr"uckungen zu erhalten (benötigt eventuell columns=flexible)
   keepspaces=true,
% Formatierung der Schl"usselwörter
   keywordstyle=\bfseries\ttfamily\color{codepurple},
% Wohin mit den Zeilennummern?
% mögliche Werte sind: none, left, right
   numbers=left,
% Formatierung der Zeilennummern
   numberstyle=\tiny\color{codegreen},
% Abstand zwischen Zeilennummern und Quelltext
   numbersep=5pt,
% nummeriert nur jede i-te Zeile
   stepnumber=1,
% Zeige alle Leerzeichen mit Hilfe von speziellen Unterstrichen;
% "uberschreibt 'showstringspaces'
   showspaces=false,
% unterstreiche Leerzeichen nur in Zeichenketten
   showstringspaces=false,
% Zeige Tabulatoren in Zeichenketten mit Hilfe von speziellen
% Unterstrichen
   showtabs=false,
   flexiblecolumns=false,
% Schrittweite zwischen Zeilennummern.
% Der Wert 1 bedeutet, dass jede Zeile nummeriert wird.
   tabsize=1,
% Formatierung der Zeichenketten.
   stringstyle=\color{orange}\ttfamily,
% leere Zeilen werden nicht nummeriert
   numberblanklines=false,
% Abstand zum linken Seitenrand
   xleftmargin=1.2em,
% Abstand zum rechten Seitenrand
   xrightmargin=0.4em,
   aboveskip=2ex,
}

\lstdefinestyle{py}{%
   language=Python,
}
\lstdefinestyle{hs}{%
   language=Haskell,
}
\lstdefinestyle{tex}{%
    language=[LaTeX]TeX,
% Um LaTex im Quelltext einzuf"ugen
    escapeinside={\%*}{*)},
% Hervorhebung der TeX-Schl"usselwörter
    texcsstyle=*\bfseries\color{blue},
    morekeywords={*,$,\{,\},\[,\],lstinputlisting,includegraphics,%$
        rowcolor,columncolor,listoffigures,lstlistoflistings,
        subsection,subsubsection,textcolor,tableofcontents,colorbox,
        fcolorbox,definecolor,cellcolor,url,linktocpage,subtitle,
        subject,maketitle,usetikzlibrary,node,path,addbibresource,
        printbibliography},
    numbers=none,
    numbersep=0pt,
    xleftmargin=0.4em,
}

\lstdefinestyle{java}{%
    language=Java,
    extendedchars=true,
}

% f"uge einen x64-Assembler Dialekt hinzu
\lstdefinelanguage[x64]{Assembler}
% basiert auf dem "x86masm" Dialekt
   [x86masm]{Assembler}
 % zusätzliche Schl"usselwörter
   {morekeywords={CDQE,CQO,CMPSQ,CMPXCHG16B,JRCXZ,LODSQ,MOVSXD, %
                  POPFQ,PUSHFQ,SCASQ,STOSQ,IRETQ,RDTSCP,SWAPGS, %
                  rax,rdx,rcx,rbx,rsi,rdi,rsp,rbp, %
                  r8,r8d,r8w,r8b,r9,r9d,r9w,r9b}
}

\lstdefinestyle{c}{
    language=c,
    extendedchars=true,
}

% //////////////////// eigene Anweisungen ////////////////////
% benötigt package xspace
\newcommand\FU{Freie Universit\ät Berlin\xspace}
\newcommand\gdw{g.\,d.\,w.\xspace}
\newcommand\oBdA{o.\,B.\,d.\,A.\xspace}
\newcommand\N{\mathbb{N}\xspace}
\newcommand\Q{\mathbb{Q}\xspace}
\newcommand\R{\mathbb{R}\xspace}
\newcommand\Z{\mathbb{Z}\xspace}
\newcommand\ohneNull{\ensuremath{\backslash\lbrace 0\rbrace}}% \{0}
\newcommand\ohne[1]{\ensuremath{\backslash\lbrace#1\rbrace}}% \{#1}
% Schreibt Befehl \dh in \dhALT um
\let\dhALT\dh
%renew "uberschreibt command \dh
\renewcommand\dh{d.\,h.\xspace}
\newcommand{\from}{\ensuremath{\colon}}
\newcommand{\floor}[1]{\lfloor{#1}\rfloor}
\newcommand{\ceil}[1]{\lceil{#1}\rceil}
\newcommand{\abbrev}[2]{\expandafter\newcommand\csname
                        #1\endcsname{#2\xspace}}
\newcommand{\cclasss}[2]{\abbrev{#1}{\textsf{#2}}}
\newcommand{\cclass}[1]{\cclasss{#1}{#1}}
\cclasss{ccP}{P}
\cclass{NP}
\cclass{LOGSPACE}
\cclass{NL}

% //////////////// mathematische Funktionen ////////////////////
\DeclareMathOperator{\True}{True}
\DeclareMathOperator{\False}{False}

% //////////////////// eigene Theoreme ////////////////////
\newtheorem{theorem}{Satz}
\newtheorem{corollary}[theorem]{Korollar}
\newtheorem{lemma}[theorem]{Lemma}
\newtheorem{observation}[theorem]{Beobachtung}
\newtheorem{definition}[theorem]{Definition}
\newtheorem{Literatur}[theorem]{Literatur}

% konfiguriert proof
\makeatletter
\newenvironment{Proof}[1][\proofname]{\par
  \pushQED{\qed}%
  \normalfont \topsep6\p@\@plus6\p@\relax
  \trivlist
  \item[\hskip\labelsep
        \bfseries
    #1\@addpunct{.}]\ignorespaces
}{%
  \popQED\endtrivlist\@endpefalse
}
\makeatother

% //////////////////// eigene Farben ////////////////////
\let\definecolor=\xdefinecolor
\definecolor{FUgreen}{RGB}{153,204,0}
\definecolor{FUblue}{RGB}{0,51,102}

\definecolor{middlegray}{rgb}{0.5,0.5,0.5}
\definecolor{lightgray}{rgb}{0.8,0.8,0.8}
\definecolor{orange}{rgb}{0.8,0.3,0.3}
\definecolor{azur}{rgb}{0,0.7,1}
\definecolor{yac}{rgb}{0.6,0.6,0.1}
\definecolor{Pink}{rgb}{1,0,0.6}

\definecolor{bgcolour}{rgb}{0.97,0.97,0.97}
\definecolor{codegreen}{rgb}{0,0.6,0}
\definecolor{codegray}{rgb}{0.35,0.35,0.35}
\definecolor{codepurple}{rgb}{0.58,0,0.82}
\definecolor{codeblue}{rgb}{0.4,0.5,1}

% //////////////////// eigene Einstellungen ////////////////////

% verhindert Einr"uckung der 1. Zeile eines Absatzes
\parindent 0pt
%
% <-- Name der "ubungsleitung eintragen
\newcommand{\tutor}{Serkan Süner}
% <-- Nummer im KVV nachschauen
\newcommand{\tutoriumNo}{4}
% <-- Nummer des "ubungszettels
\newcommand{\ubungNo}{9}
% <-- Name der Lehrveranstaltung eintragen
\newcommand{\veranstaltung}{Mafi 3}
% <-- z.B. SoSo 17, WiSe 17/18
\newcommand{\semester}{WiSe 17/18}
% <-- Hier die Namen eintragen
\newcommand{\studenten}{Merlin Joseph \& Anton Oehler}
% <-- Hier Anzahl der Aufgaben eintragen
\newcommand{\aufgNo}{4}

% /////////////////////// BEGIN DOKUMENT /////////////////////////
\begin{document}
% ////////////// Bepunktung //////////////
%linksb"undig
\makebox[\dimexpr\textwidth][l]{%
    \begin{minipage}{\linewidth}
        \newcounter{AufgNo}
        \setcounter{AufgNo}{\aufgNo}
        \stepcounter{AufgNo}   % AufgNo++
        \newcounter{zahl}
        \def\and{&\xspace}
        \renewcommand{\arraystretch}{1.3}\setlength{\tabcolsep}{1em}
        \begin{tabular}{*{\value{AufgNo}}{|c} |}
            \hline
             \setcounter{zahl}{1}
             \whiledo{\value{zahl} < \value{AufgNo}}{%\AufgNo
                 \thezahl\and\stepcounter{zahl}%
             } $\sum$ \\ \hline
             \setcounter{zahl}{1}
             \whiledo{\value{zahl} < \value{AufgNo}}{%\AufgNo
                 \phantom{X}\and\stepcounter{zahl}%
            } \phantom{X}\\ \hline
        \end{tabular}
    \end{minipage}
}
% ////////////// Daten //////////////
\begin{center}

{\Large \veranstaltung, \semester}\par
{\large Tutor\_in: \tutor, Tutorium \tutoriumNo}\par
{\Large \"Ubung \ubungNo}\par
{\large \studenten}\par
\today
\end{center}
\vspace{-3ex}             % Abstand
\rule{\linewidth}{0.8pt}  % horizontale Linie

% /////////////////////// Aufgabe 1 /////////////////////////
\section*{Hauptsatz der Differential- und Integralrechnung}

Integrationsregeln:

Für zwei beliebige $f, g: [a,b] \rightarrow \R$ beschänkt und stückweise
stetig gilt:

\begin{enumerate}
\item \[
    \int_a^b \alpha f(x) + \beta g(x) dx = \alpha\int_a^b f(x) dx + \beta \int_a^b g(x) dx
\]
\item \[
    \forall c \in (a,b) \int_a^b f(x) dx = \int_a^c f(x) dx + \int_c^b f(x) dx
\]
\item \[
    \text{Ist } f(x) \leq g(x) \text{ für alle } x \in [a,b] \Rightarrow
        \int_a^b f(x) dx \leq \int_a^b g(x) dx
\]
\end{enumerate}

Folgerungen:
\begin{enumerate}
\item $f: [a,b] \to \R$ beschränkt und stückweise stetig.

    m,M minimaler/maximaler Funktionswert von f.

    \[m(b-a) \leq \int_a^b f(x) dx \leq M(b-a)\]

\item \[\left|\int_a^b f(x) dx\right| \leq \int_a^b |f(x)| dx\]
\item
    \[-|f(x)| \leq f(x) \leq |f(x)|\]
    \[\int_a^b -|f(x)| \leq \int_a^b f(x) \leq \int_a^b |f(x)|\]
\end{enumerate}

Mittelwertsatz (der Integralrechnung):

Sind $f,g: [a,b] \to\R$ stetig auf $[a,b]$ und
$g(x) \geq 0$ für alle $x \in [a,b]$, dann gibt es
ein $\xi\in [a,b]$, so dass
\[\int_a^b f(x) \cdot g(x) dx = f(\xi) \cdot \int_a^b g(x) dx\]

Sei $m$ bzw. $M$ minimaler/maximaler Funktionswert von $f$ über $[a,b]$
\[\forall x\in [a,b]\quad m\>g(x) \leq f(x) g(x) \leq M\>g(x)\]
Regel 3: \[m\> \int_a^b g(x) dx \leq \int_a^b g(x) dx \leq M \int_a^b g(x) dx\]
\[\Rightarrow \int_a^b f(x) g(x) dx = \widetilde{m} \int_a^b g(x) dx
\text{ für ein } m \leq \widetilde{m} \leq M
\]
Zwischenwertsatz (für stetige Funktionen):
\[\exists \xi \in [a,b] \quad f(\xi) = \widetilde{m}\]

Hauptsatz: $f: I \to \R$ stetig auf offenem Intervall I

$a,b \in I \quad a \leq b$

\begin{enumerate}
\item Die Funktion $F_a(x) := \int_a^x f(x) dx$ ist Stammfunktion von $f$ auf $I$
\item Für eine beliebige Stammfunktion $F$ von $f$ und den Ausdruck
$F(x) |_a^b = [F(x)]_a^b := F(b) - F(a)$ gilt:
\[\int_a^b f(x) dx = F(x) |_a^b\]
\end{enumerate}
Beweisidee: zu zeigen
\[\lim_{h\to 0} \dfrac{F_a(x+h)-F_a(x)}{h} = f(x)\]
Betrachte \[\underbrace{F_a(x+h)-F_a(x)}_{\approx h \cdot f(x)} \text{ (und }h>0\text{)}\]

formal:
\begin{align*}
    F_a(x+h) - F_a(x) &= \int_a^{x+h} f(x) dx - \int_a^x f(x) dx\\
    &= \int_x^{x+h} \underbrace{1}_{=g(x)}\cdot f(x) dx = \underbrace{f(\xi)}_{\xi\in [x,x+h]} \cdot \int_x^{x+h} 1 dx = f(\xi) \cdot h\\
\end{align*}
\[\dfrac{F_a(x+h)-F_a(x)}{h} = f(\xi)\]
Wenn $h\to 0$, dann $\xi\to x$ und $f(\xi) \to f(x)$ weil $f$ stetig
$\Rightarrow F'_a(x) = f(x)$

% /////////////////////// END DOKUMENT /////////////////////////
\end{document}

